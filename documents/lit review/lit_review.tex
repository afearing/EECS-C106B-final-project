\documentclass[conference]{IEEEtran}
\RequirePackage{cite}
\RequirePackage{amsmath,amssymb,amsfonts}
\RequirePackage{algorithmic}
\RequirePackage{graphicx}
\RequirePackage{textcomp}
\RequirePackage{xcolor}
\RequirePackage{hyperref}
\def\BibTeX{{\rm B\kern-.05em{\sc i\kern-.025em b}\kern-.08em
    T\kern-.1667em\lower.7ex\hbox{E}\kern-.125emX}}
\begin{document}

\title{Literature Review}
\author{Andrew~Fearing, Neelay~Junnarkar,  Hamza~Kamran~Khawaja}
\maketitle

\begin{abstract}
This document is for writing about the papers we find. Currently the plan is to do control of boats.
\end{abstract}

\section{Questions}
We want lane-keeping for boats. Are we doing sailboats or motorboats?
\section{Helpful search terms}
\begin{itemize}
    \item USV (unmanned surface vehicles)
    
\end{itemize}

\section{Papers}
\subsection{Fuzzy Guidance Controller for an Autonomous Boat}
\cite{Vaneck1997} An old paper on waypoint following.


\subsection{Dynamic simulation model for an autonomous sailboat}
\cite{Buehler2018} This paper derives a dynamic model for a sailboat. There is a \href{https://github.com/simonkohaut/stda-sailboat-simulator/tree/master/src}{GitHub repo} with their Python simulation.


\section{Control}
\cite{nan}
\section{Navigation}

\section{Models}
\cite{Huang2017} Tries to figure out how close a boat model is from reality. Inconclusive, they only do preliminary work. Further work could be done, but it probably requires a real life boat.

\cite{Setiawan2020} Tries to make a dynamic model of autonomous sailboat.
\bibliographystyle{IEEEtran}
\bibliography{refs.bib}

\end{document}

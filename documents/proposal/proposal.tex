\documentclass[conference]{IEEEtran}
\RequirePackage{cite}
\RequirePackage{amsmath,amssymb,amsfonts}
\RequirePackage{algorithmic}
\RequirePackage{graphicx}
\RequirePackage{textcomp}
\RequirePackage{xcolor}
\RequirePackage{hyperref}
\RequirePackage{csquotes}
\setkeys{Gin}{width=0.4\textwidth}
\def\BibTeX{{\rm B\kern-.05em{\sc i\kern-.025em b}\kern-.08em
    T\kern-.1667em\lower.7ex\hbox{E}\kern-.125emX}}
\begin{document}
\title{EECS 106B Final Project Proposal}
\author{Andrew~Fearing, Neelay~Junnarkar,  Hamza~Kamran~Khawaja}
\maketitle


\begin{abstract}
We will develop a robust control scheme for sailboats (underactuated USV's ???) to navigate channels.
\end{abstract}


\section{About Us}
Neelay is a 4th year undergraduate in EECS, and has taken courses such as EE 128, 221A/222, and (of course) EECS 106A. His interests are primarily related to control and motion planning.

Andrew is a 4th year undergraduate in EECS. He has taken EE 120, EE C128, and EECS C106A. His interests in robotics are in micro-scale mobile robots.

Hamza is a 4th year undergraduate in EECS. He has taken EECS106A,and EECS126. His interests in robotics are in machine learning/ artificial intelligence and dynamically responsive, maneuverable robotic systems with a spice of efficiency/optimization.
\section{Research Question}
Can we control a sailboat so that it keeps its lane? Autonomously navigate a canal.
\section{Motivation}
Boat stuck. Existing systems to survey and navigate the ocean surface include fixed monitoring systems, that either drift (floating buoys) or are fixed (moored data-buoys), survey ships and satellites. The floating buoys are easily lost, the moored buoys are expensive to deploy and maintain, survey ships are limited in range, and satellites take a long time to manufacture with expensive cost and lack the resolution and accuracy of in-situ devices \cite{Sauzé2006}.
\begin{figure}
    \centering
    \includegraphics{documents/proposal/Suez_Canal_blocked_by_Ever_Given_March_27_2021.jpg}
    \caption{Boat stuck\label{fig:boat_stuck}}
\end{figure}
Control and safety of automobiles is an area of intense research focus. We want to try applying the same techniques to a different system: autonomous boats, or in engineering terms, \enquote{Unmanned Surface Vehicles}. In particular, we will be looking at autonomous sailboats, since \textbf{TODO}.

\section{Proposed Methodology}
We plan to use simulation to quantify the performance of our control scheme. We will examine literature in USV's.
We will simulate 

We will need to carefully define the scope of our project. At the moment, it is not clear how high the fidelity of our simulation will need to be 
\section{Related Work}
There is already work in the area of of USV's. Current work has focused on robust control for motorboats.

\cite{Setiawan2020} and \cite{Buehler2018} develop a dynamic models of an autonomous sailboats, though they do not go into the application of controls.

We hope to incorporate the control barrier functions we learned about in our presentation on \cite{Ames2019} to provide robustness.


\section{Experimental Plan}
Using a dynamic model as that proposed in \cite{Buehler2018}, we will simulate and analyze the performance of our control scheme.

\bibliographystyle{IEEEtran}
\bibliography{refs.bib}
\end{document}
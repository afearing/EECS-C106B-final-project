\documentclass[journal]{IEEEtran}
% \RequirePackage[utf8]{inputenc}
\RequirePackage{hyperref}
% \RequirePackage{caption}
\RequirePackage{cite}% use this because it works properly with IEEE
\RequirePackage{amsmath}
\RequirePackage{amsfonts}
\RequirePackage{amssymb}
\RequirePackage{physics}
\RequirePackage{siunitx}
\RequirePackage{url}
\RequirePackage{listings}
\RequirePackage{graphicx}
\usepackage{caption}
\usepackage{subcaption}
\usepackage[ruled,vlined]{algorithm2e}

\setkeys{Gin}{width=3in}
\graphicspath{{./figures/}}
\RequirePackage{bm}
\RequirePackage{float}
\RequirePackage{cancel}
\usepackage[normalem]{ulem}


% \RequirePackage[framed,numbered,autolinebreaks,useliterate]{mcode}
\RequirePackage{metalogo} % \LaTeX logo

\begin{document}
\title{EECS 106B Final Project Proposal}
\author{Andrew~Fearing, Neelay~Junnarkar,  Hamza~Kamran~Khawaja}
\maketitle

% argument is your BibTeX string definitions and bibliography database(s)
\bibliographystyle{IEEEtran}
\bibliography{biblio.bib}

\begin{abstract} In this paper, we investigate algorithms that perform two-view and multi-view feature matching and reconstruction. The algorithms, namely the eight-point algorithm, the factorization algorithm, and the two-view triangulation as in \cite{ma2012invitation}. Data points from image sets depicting a city and house are used. The reconstructions after applying the algorithms are re-projected onto the original image/model and a re-projection error is calculated to evaluate performance. We found best performance with the two-view 8-point algorithm with a re-projection error of 4, while the four-view reconstruction had an error of 1624. In other words, least error was observed in two-view reconstruction, while most error was observed in multi-view reconstruction.
\end{abstract}


\section{About Us}
Neelay is a 4th year undergraduate in EECS, and has taken courses such as EE 128, 221A/222, and (of course) EECS 106A. His interests are primarily related to control and motion planning.

\section{Research Question}

\section{Motivation}

\section{Proposed Methodology}

\section{Related Work}

\section{Experimental Plan}

\end{document}